\chapter{Introduction}
\section{Motivation}
{\Large
 \begin{itemize}
  \item Small vertical beam size ({\color{red} goal 1})
  \begin{itemize}
   \item Achieve $\sim 37$nm
   \item Validate Local chromaticity correction
  \end{itemize}
  \item Stabilization of beam center ({\color{blue} goal 2})
  \begin{itemize}
   \item down to $\sim2$nm
  \end{itemize}
 \end{itemize}
}
 $\,$
  \includegraphics[angle=0,scale=0.35]{scalefactors.jpg}
{\LARGE
  \begin{itemize}
   \item Locate BPMs to enable the \textbf{ maximum possible} beam position resolution
   \item Precision $\sim 5\mu$m
   \item Calibration $\sim 10^{-4}$
  \end{itemize}
 \hspace*{5cm}$\Downarrow$\\
 Displace each BPM block \textbf{independently}
 }\par
\includegraphics[angle=0,scale=0.2]{ATF2layout33.jpg}
% \end{frame}
% \begin{frame}
\hspace{1cm}
\includegraphics[angle=0,scale=0.16]{chambrevide.jpg}\\
\hspace*{1cm}\includegraphics[angle=0,scale=0.22]{BPMs01.jpg}\hspace*{0.2cm}
\includegraphics[angle=0,scale=0.0355]{IMAG0460.jpg}\par
\includegraphics[angle=0,height=7.5cm,width=12cm]{interface.jpg}

\section{Alignment Requirements}
Consider Figure \ref{beamdist}, where vertical position histogram is to be filled with each passing bunch. The scale is divided in $N$ parts, each one with same width. It also has a minimum and maximum values. As the beam passes, the cavity signal is going to be ideally proportional to beam centroid position $y_k$, where $k$ stands for the $k$-th bunch, all charge is concentrated there and the distribution effect is a multiplying factor. In red, it is possible to see that beam distribution might change slightly, but in general is going to be gaussian and to have same size (beam size = 1$\sigma_y$).
\begin{figure}[htb]
\begin{center}
 \includegraphics[angle=0,scale=0.1]{DSC_0026.jpg}\caption{Beam distr.}\label{beamdist}
 \end{center}
\end{figure}
Each vertical measurement $y_k$ is displaced from BPM center in an amount determined by a vertical offset $y_0$ with respect to cavity center and a random number proportional to beamsize ($y_j=p\sigma_y$, and $p$ between 10 to 20\%, jitter). Therefore, from a sample $Y$ of $m$ bunches measured, $y_0=\langle Y\rangle$ and $y_j=\langle Y^2\rangle-\langle Y\rangle^2$. Jitter will also be present in horizontal position and angles. For the moment, only discretisation level will be the only error source contributing to each $y_k$.\par
The objective is to measure $\boldsymbol{y_j}$ with $\boldsymbol{10^{-4}}$ relative or \textbf{1nm} precision.\par
The preceeding statament means that the scale division used to measure $y$ should comply with\par
\begin{equation}
  \frac{y_j}{N}\leq1\text{nm, or, }\frac{1}{N}\leq10^{-4}
\end{equation}
In order to satisfy both, $N>10^{4}$. It also sets two ranges for the jitter measurement:
\begin{itemize}
 \item $y_j<10\mu$m, where resolution is below 1nm.
 \item $y_j\geq10\mu$m, where resolution is $10^{-4}$ maximum.
\end{itemize}
The maximum number counts used to measure jitter will always be below 10000. If we consider jitter conviniently centered, then it will be 5000 counts below $N/2$ and 5000 counts above $N/2$. The allowed offset under this circumstances, will be from 5000 counts to $N-5000$ counts. In total we have substracted 10000 counts from the total range, and the maximum offset will be $N-10000$ counts.\par
The number of counts $N$ comes from the binary $2^b$ discretisation scale where $b$ is the number of bits. As oscilloscopes are only 8 bits, the max. resolution will be $10^{-2}$. For a system with 14 bits, $N=16384$,  6384 counts are free for tolerances to different effects including offset. Any offset below $6.384\mu$m will will allow 1nm precision. Above this size, only $10^{-4}$ might be reached.\par
Table \ref{toletab} shows the different level of tolerances corresponding to different precision levels for 1nm resolution and 14 bits discretisation.\par
\begin{table}[hbt]
\begin{center}
 \begin{tabular}{|c|c|c|}\hline
 Precision & Total OFFSET SIGNAL ($\mu$m) & Centered ($\mu$m)\\\hline
 $10^{-2}$ &16.824&$\pm$8.142 \\\hline
 $10^{-3}$ &15.384&$\pm$7.692\\\hline
 $10^{-4}$ &6.384&$\pm$3.192\\\hline
 \end{tabular}
 \caption{Tolerances and precision.}\label{toletab}
 \end{center}
\end{table}
Consider now the signal $S$ as the sum of the two outputs from the cavity (which are ideally 180$\deg$ separated) as seen in figure \ref{Ssignal},\par
\begin{figure}[htb]
 \begin{center}
  \includegraphics[angle=0,scale=0.4]{electr2.jpg}\caption{Signal S.}\label{Ssignal}
 \end{center}
\end{figure}
Signal $S$ for the vertical outputs of one BPM and one bunch is composed of ($k$ index is not used):
\begin{equation}
S = y+is_p\theta_p+s_{xy}(x+is_y\theta_y)+x\theta_r
\end{equation}
where $i$ is the imaginary number indicating a 90 degrees phase difference (angle signals have 90 phase difference with respect to position signals from cavity). $s_p$ is the sensitivity to angle, $s_{xy}$ is the inverse of X-Y isolation, $s_y$ is the yaw angle sensitivity. $x,y,\theta_p,\theta_r,\theta_y$, are horizontal position, vertical position, pitch, roll and yaw angles BPM with respect to beam. Not only vertical position contributes to $S$.\par
Signal $S$ can be separated in real and imaginary parts:
\begin{equation}
S = (y+s_{xy}x+x\theta_r)+i(s_p\theta_p+s_{xy}s_y\theta_y)
\end{equation}
This signals are rotatated by an arbitrary angle $\phi$ to obtain the $I'$ and $Q'$.
\begin{align*}
S &= \underbrace{(y+s_{xy}x+x\theta_r)(\cos\phi+i\sin\phi)}+\underbrace{i(s_p\theta_p+s_{xy}s_y\theta_y)(\cos\phi+i\sin\phi)}\\
  &= I' + Q' 
\end{align*}
In the case of perfect IQ rotation ($\phi=0$), all imaginary (angle and others) component is removed from real (position) component in the $S$ signal. However, in practice this rotation could be achieved to precision set by $\Delta\phi$, then
\begin{equation}
 S = [(y+s_{xy}x+x\theta_r)+\Delta\phi(s_p\theta_p+s_{xy}s_y\theta_y)]+i[\Delta\phi(y+s_{xy}x+x\theta_r)+(s_p\theta_p+s_{xy}s_y\theta_y)]
\end{equation}
where all term have been considered positive as they represent tolerances (position errors). At this point will be only interested in the real part as it contains most of the vertical position signal.
\begin{equation}
 S = (y+s_{xy}x+x\theta_r)+\Delta\phi(s_p\theta_p+s_{xy}s_y\theta_y)
\end{equation}
The last equation shows the contribution to vertical signal from the relative position BPM to beam.\par
Isolation X-Y (1/$s_{xy}$) was measured to be under 50dB (Pin/Pout$<10^{-5}$), sensitivity to pitch ($s_p$) was measured to be $3.2\mu$m/mrad, and sensitivity to yaw ($s_y$) is 2.9$\mu$m/mrad estimated in similar way to $s_p$. Precision on phase rotation ($\Delta\phi$) is still to be determined.\par
From this point, it will be assumed that mechanical positions could be measured within an absolute error shown in Table \ref{mechprec}.\par
\begin{table}[htb]
\begin{center}
 \begin{tabular}{|c|c|}\hline
  Axis & Mechanical Precision ($\mu$m)\\\hline
  Vertical & 1\\
  Horizontal & 5\\
  Longitudinal & 5 \\\hline
 \end{tabular}\caption{Position mechanical precision}\label{mechprec}
 \end{center}
\end{table}
Assuming that angles can be estimated from two coplanar points in the BPM, then, minimum resolvable angle might be within the values shown in Table \ref{angleprec}. Figure \ref{PAcontrol} shows the approximate location of angle control points.\par
\begin{table}[htb]
 \begin{center}
  \begin{tabular}{|c|c|c|}\hline
  Angle (Symbol) & Estimated min. resolvable value (mrad) & Distance between points (mm)\\\hline
   Pitch ($\theta_p$)&0.042&120\\
   Yaw ($\theta_y$)&0.042&120\\
   Roll ($\theta_r$)&0.033&30\\\hline
  \end{tabular}\caption{Angle mechanical precision}\label{angleprec}
 \end{center}
\end{table}
\begin{figure}[htb]
 \begin{center}
  \includegraphics[angle=0,scale=0.4]{angles.jpg}\caption{Points of position and angle control.}\label{PAcontrol}
 \end{center}
\end{figure}
Using the nominal optics 1BX1BY, there are two relevant cases:
\begin{itemize}
 \item Largest beam size: it is at IPBPMA, having 57.657$\mu$m of vertical beam size. Jitter will be around 11.531$\mu$m (20\%)and 5.766$\mu$m (10\%). Both cases are close or inside of the 1nm resolution.
 \item Beam waist at the IP: in this case vertical beams size is 37$\sim$54nm, and beam jitter will be around 4$\sim$10nm. It is clearly inside the 1nm resolution case, however, SNR (Signal to Noise Ratio) must be considered in order to be able to separate jitter from other sources.
\end{itemize}
