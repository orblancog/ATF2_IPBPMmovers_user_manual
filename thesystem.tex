\chapter{The System}
\section{System Description}
At the ATF2 IP, inside the vacuum chamber, the BPMs positioning system has been installed during the first two weeks of July 2013.
It will move two blocks:
\begin{itemize}
\item BLOCK BPMs AB
\item BLOCK BPM C
\end{itemize}
This movement has two degrees of freedom: vertical (y) and lateral (x), both transverse to the beam direction. Beam crosses BPMAB first, then and empty space where Shintake monitor laser will be aligned, then crosses BPM3 and goes out of the vacuum chamber.

These two blocks of BPMs, moving in two directions each, are displaced by Piezo-Movers (or just movers) made by different companies. German «PI electronics« moves  BPMC and French «Cedrat Technologies« moves BPMAB.

Each block has 3 vertical and one lateral movers. Lateral is underneath. Four movers are used per block, eight in total.

\section{The Piezo Movers}
Piezo mover changes its position as a function of voltage. Each one of the eight movers has its own control electronics block composed of: 
\begin{itemize}
 \item the mover
\item the strain gauge
\item manufacturer module to:
\begin{itemize}
 \item set piezo mover voltage (high voltage)
 \begin{itemize}
\item PI module E-621
\item CEDRAT module LA75
\end{itemize}
\item read strain gauge
\begin{itemize}
\item PI module E-621 (same as control)
\item CEDRAT module SG75
\end{itemize}
\item set feedback operation (ON/OFF)
\item set control mode (external control from PLC is used)
\end{itemize}
\item PLC channel to
\begin{itemize}
\item set control voltage values equivalent to piezo mover displacement
\item read voltage values equivalent to strain gauge deformation
\end{itemize}
\end{itemize}

\subsection{Mover range}
PI and Cedrat have different ranges: PI (300um), Cedrat (250um)\par

Control voltage and its relation with displacement (min-max) varies between both companies:\par
\begin{itemize}
\item PI: 		(min)   0V   	to 	(max) 10V
\item CEDRAT: 	(max) -1V 	to	(min)    7V
\end{itemize}
It is important to note that:
\begin{itemize}
\item The voltage-displacement relation is inverse for CEDRAT movers.
\item When the system is OFF (0V), PI mover are at its minimum, however CEDRAT are not.
\end{itemize}

\subsection{Feedback and NO Feedback}
There are two operation possibilities per mover: Feedback (fb) and No Feedback (no fb). It means 8 feedback loops. On each case the control module sets a voltage value on the mover and read the strain gauge to create a closed loop. However, their implementations are different on each company.

PI has an analogue feedback system adjusted to the best possible performance on an individual mover (avoid piezo-mover channel interchange). Do not exchange fb/no fb while the equipment is ON. Turn it OFF before any modification. 
Feedback is controlled by the switch number 3  on the front panel. 
Feedback ON (left) / Feedback OFF (right)
During operation, when feedback is ON and working properly two green LED are on (Power LED, On Target LED). When no feedback just the Power LED is green.

Cedrat has a digital feedback system with PID parameters per mover. Avoid piezo-mover channel interchange. Feedback could be turn off/on by two means:
Switch SERVO ON in the front pannel. Feedback ON(right)/Feedback OFF (left)
Using sofware  HDPM45V16. To activate or desactivate permanently the fb.

\subsection{Mover Control}
Control per mover\par
\begin{center}
\begin{tikzpicture}
 \node (img1) {\includegraphics[scale=0.5]{Page1.pdf}};
 %\node (img2) at (4.4,1.5) {\includegraphics[scale=0.1,angle=-90]{imagestep12.pdf}};
%  \node (img3) at (4.4,2.6) {$\sigma_y=$?};
 %\node (img3) at (4.4,1.5) {{\LARGE\color{red} ?}};
\end{tikzpicture}
\end{center}
{\tiny BLOCK AB 1mV=31.25nm, BLOCK C 1mV=30nm}\par

\section{The PLC}
Two NI9263 are used to set analogue voltage into the mover control electronics.\par
Two NI9239 are used to read analogue voltage from the strain gauge readback.\par
One NI9219 is used to read temperature.\par
These modules are connected to the chassis NI9188 which is connected by network to a working station with Labview installed.
The block chassis+NI modules is called \textbf{PLC}.\par

National instruments Chassis: NI9188
\begin{itemize}
\item Mac Address: 0080.2f14.b777
\item DHCP
\item IP address (ATF) 31.1.1.39
\item IP address (KEK): None
\item Host name: ipmv-plc.ip-local
\item Net Mask: 255.255.255.0
\item connected to ip-local during installation
\item Chassis NI9188 can connect up to 8 modules, not all are used:
\begin{itemize}
\item PI
\begin{itemize}
\item Module 5: NI9263 Digital to analogue converter
\item Module 6: NI9239 Analogue to digital converter
\end{itemize}
\item Cedrat
\begin{itemize}
\item Module 2: NI9263 Digital to analogue converter
\item Module 3: NI9239 Analogue to digital converter
\end{itemize}
\item Temperature
\begin{itemize}
\item Module 8: NI9219 Temperature probes
\begin{itemize}
\item Cedrat: 	Channel 0
\item PI: Channel 2
\end{itemize}
\end{itemize}
\end{itemize}
\end{itemize}

\section{The PC}
It is used to control from close locations the BPM positioning system. It sets the digital values to put in the PLC and reads the digital values from the PLC channels corresponding to strain gauges.\par

\subsection{Used Software}
Windows 7 Français\par
National Instrument - Labview 2011\par
National Instruments - Measurement $\&$s Automation Explorer (NI MAX)\par
Gimp 2.8\par
Microsoft office – Excel 2013\par
cmd\par
Highly Dynamic and Precise motion 45 V 1.6 (HDPM45V16)– Cedrat Technologies\par
LAL Computer (Laptop)\par
Account: **********(not in this document)\par
Password: *********(not in this document)\par
Processor Inter Core i7 vPro\par
Mac Address: d067.e550.620e\par
DHCP\par
IP address(ATF): 31.1.1.38\par
IP address (KEK): Wifi connection (MAC registered at KEK)\par
Net Mask: 255.255.255.0\par
connected to ip-local during installation\par
Folder Content description\par
Path to applications and info\par
Bureau/Actionneurs Piezo/Applis\par
All applications were done in Labview. Title gives a hint of its content:\par
	(keywords used in titles):
oscilloscope: uses de ADC to read signals\par
generateur: function generator, uses de DACs to produce signals\par
Actionneurs positionnement BPMs: activate the system displacement\par
Ethernet: connected by wired network,\par
USB: corresponds to previous version connected by USB.\par
Verticaux groupe: all 3 vertical mover movers are activated by one voltage control\par
 mouvements identiques: first version of BPM displacement system (PI and CEDRAT) integrated.\par
Jauges: stores the strain gauges info in excel format.\par
temp: stores temperature info in excel format.\par
epics: control from epics system, Labview works as interface.\par
Actionneurs multicycles: several cycles over the defined voltage range are performed\par
Cedrat, PI: identifies the group of movers to use.\par
Vertical, lateral fixe: it means that one direction of movement is set (fixed) to a voltage value while the other direction varies in cycles.\par

\section{The BPMs}
\subsection{Coordinate system}\par
\centering
Each BPM has its own coordinates with respect to a reference system centered \textbf{electrically} and aligned with the beam
%\begin{columns}[t]
%\begin{column}{3.5cm}
  \includegraphics[scale=0.3,angle=0]{fig23.pdf}\\
  {\color{red}Beam}, {\color{forestgreen} BPMs}\\
%\end{column}
%\begin{column}{7cm}
\vspace{-3.3cm}
% \textbf{Variables}
\begin{itemize}
 \item Beam Position
 \\$x_A,y_A,z_A$\\$x_B,y_B,z_B$\\$x_C,y_C,z_C$
 \item BPM Angles respect to ref. system\\
 $\theta_{Ap},\theta_{Ar},\theta_{Ay}$\\
 $\theta_{Bp},\theta_{Br},\theta_{By}$\\
 $\theta_{Cp},\theta_{Cr},\theta_{Cy}$
\end{itemize}
%\end{column}
%\end{columns}
\centering
{\tiny All systems relate to a common \textbf{mechanical} reference system, no rotations, just translations}\\
\includegraphics[scale=0.2,angle=0]{fig17.pdf}\\\centering
{\tiny \textbf{One of the BPMs reference system could be chosen to coincide with the common}\\
% \\Ideally just translations in the longitudinal direction
}
% \\

%\begin{frame}{Movers}\centering
{\tiny There is a set of movers to control BPM position}\vspace{-0.5cm}
\centering
%\begin{columns}
%\begin{column}{3.5cm}
\includegraphics[scale=0.30,angle=0]{fig25.pdf} 
%\end{column}
%\begin{column}{5.5cm}
\begin{align*}
 x &= x_0+f_x(M_{0,B})\\
 y &= y_0+f_y(M_{123,CDE})\\
 z &= z_0\\
 \theta_p &= \theta_{p0}+f_p(M_{123,CDE})\\
 \theta_r &= \theta_{r0}\\
 \theta_y &= \theta_{y0}
\end{align*}\par
\vspace*{-0.4cm}{\tiny All initial values are set during the IP BPMs installation}
%\end{column}
%\end{columns}\vspace*{0.5cm}
%\begin{columns}
% \begin{column}{5cm}
\includegraphics[scale=0.25,angle=0]{fig27.pdf}
% \end{column}
%\begin{column}{5cm}
\includegraphics[scale=0.25,angle=0]{fig26.pdf}
% \end{column}
%\end{columns}
% \end{frame}


\includegraphics[scale=0.3,angle=0]{pos_beta2}

\subsection{Sensitivity to angle (BPMA and BPMC)}
$1000\beta_y$ optics. 51 bunches recorded per each mover position. Maximum IQ value was picked each time.\\
\centering
 \includegraphics[scale=0.3,angle=-90]{image01_sense.pdf}\\
 \begin{equation}
  \text{Sensitivity to angle BPMA}<(4.69[\text{V}]-4.44[\text{V}])\left(\frac{250[\mu\text{m}]}{8[\text{V}]}\right)\left(\frac{1}{1.6[\text{mrad}]}\right)=4.9\left[\frac{\mu\textmd{m}}{\textmd{mrad}}\right]
 \end{equation}
 For BPMC, the result was 3.2 $\mu$m/mrad from FONT group. IP-BPM meeting 2014.03.26.
\raggedright

\section{Electrical connections}
\hspace*{1.4cm}\includegraphics[angle=0,scale=0.2]{link.jpg}\par
%\hspace*{1.4cm}\includegraphics[angle=0,scale=0.2]{link.jpg}
\begin{tikzpicture}
  \node (img1) {\includegraphics[scale=0.45,angle=0]{elect01.pdf}};
 %\pause
  \node (img2) at (-4.2,-3) {\includegraphics[scale=0.039,angle=0]{ima02a.jpg}};
%   \pause
  \node (img3) at (2.2,-2.1) {\includegraphics[height=2.5cm,width=6.2cm,angle=180]{ima06a.jpg}};
%   \pause
  \node (img4) at (4.4,1.0) {\includegraphics[height=1.6cm,width=3cm,angle=0]{ima04a.jpg}};
  \node (img5) at (2.0,0.0) {\includegraphics[height=1.6cm,width=3cm,angle=0]{ima05a.jpg}};
%   \pause
  \node (img6) at (4.4,3.0) {\includegraphics[height=1.6cm,width=3cm,angle=180]{ima08a.jpg}};
  \node (img7) at (1.8,1.4) {\includegraphics[height=1.6cm,width=3cm,angle=180]{ima08a.jpg}};
  \node (img8) at (1.8,3.0) {\includegraphics[height=1.6cm,width=3cm,angle=180]{ima08a.jpg}};
%   \pause
  \node (img9) at (-3.0,0.7)  {\includegraphics[height=1.6cm,width=1.4cm,angle=0]{ima10a.jpg}};
  \node (img10) at (-3.0,3.0) {\includegraphics[height=1.6cm,width=1.4cm,angle=0]{ima10a.jpg}};
%   \pause
  \node (img11) at (-4.8,0.7)  {\includegraphics[height=1.6cm,width=1.4cm,angle=0]{ima11a.jpg}};
  \node (img12) at (-4.8,3.0) {\includegraphics[height=1.6cm,width=1.4cm,angle=0]{ima12a.jpg}};
\end{tikzpicture}